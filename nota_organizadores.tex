%Este trabalho est� licenciado sob a Licen�a Creative Commons Atribui��o-CompartilhaIgual 3.0 N�o Adaptada. Para ver uma c�pia desta licen�a, visite http://creativecommons.org/licenses/by-sa/3.0/ ou envie uma carta para Creative Commons, PO Box 1866, Mountain View, CA 94042, USA.

\chapter*{Nota dos organizadores}
\addcontentsline{toc}{chapter}{Nota dos organizadores}

Nosso objetivo � de fomentar o desenvolvimento de materiais did�ticos pela colabora��o entre professores e alunos de universidades, institutos de educa��o e demais interessados no estudo e aplica��o da �lgebra linear nos mais diversos ramos da ci�ncia e tecnologia.

Para tanto, disponibilizamos em reposit�rio p�blico GitHub (\url{https://github.com/reamat/Calculo}) todo o c�digo-fonte do material em desenvolvimento sob licen�a Creative Commons Atribui��o-CompartilhaIgual 3.0 N�o Adaptada (\href{https://creativecommons.org/licenses/by-sa/3.0/}{CC-BY-SA-3.0}). Ou seja, voc� pode copiar, redistribuir, alterar e construir um novo material para qualquer uso, inclusive comercial. Leia a licen�a para maiores informa��es.

O sucesso do projeto depende da colabora��o! Participe diretamenta da escrita dos recursos educacionais, d� sugest�es ou nos avise de erros e imprecis�es. Toda a colabora��o � bem vinda. Veja mais sobre o projeto em:
\begin{center}
  \url{https://www.ufrgs.br/reamat/AlgebraLinear}
\end{center}

\vspace{0.5cm}

Desejamos-lhe �timas colabora��es!
