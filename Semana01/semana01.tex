%Este trabalho está licenciado sob a Licença Creative Commons Atribuição-CompartilhaIgual 3.0 Não Adaptada. Para ver uma cópia desta licença, visite http://creativecommons.org/licenses/by-sa/3.0/ ou envie uma carta para Creative Commons, PO Box 1866, Mountain View, CA 94042, USA.

\documentclass[../livro.tex]{subfiles} %%DM%%Escolher document class and options article, etc

%define o diretório principal
\providecommand{\dir}{..}

%%%%%%%%%%%%%%%%%%%%%%%%%%%%%%%%%%%%%%%%%%%%%
%%%%%%%%%%%%INICIO DO DOCUMENTO%%%%%%%%%%%%%%
%%%%%%%%%%%%%%%%%%%%%%%%%%%%%%%%%%%%%%%%%%%%%

\begin{document}

%\maketitle
%\tableofcontents

% \section{Estrutura do Curso}

% Ver plano de ensino e/ou plataforma Moodle.

\chapter{Semana 1}

\section{Sistemas Lineares}

Começamos nosso estudo diretamente com um exemplo:

\begin{example}
Considere o sistema linear que consiste de duas equações e duas variáveis
\begin{equation}\label{exp1}
  \left\{
    \begin{array}{rcl}
      x+3y&=&1 \\
      2x-y&=&-2
    \end{array}
  \right..
\end{equation} Nosso objetivo é descobrir o valor (ou todos os valores) das variáveis $x$ e $y$ que satisfazem a ambas equações.

\vspace{0.2cm}

\noindent\textit{Solução 1.} Ao multiplicar toda a primeira equação por $-2$, não alteramos a solução do sistema, pois, caso quiséssemos voltar ao sistema original, bastaria dividir a primeira equação do sistema abaixo por $-2$. Ficamos assim com o sistema
\begin{equation*}
  \left\{
    \begin{array}{rcl}
      -2x-6y&=&-2 \\
      2x-y&=&-2
    \end{array}
  \right..
\end{equation*} Em seguida, somamos as duas equações e mantemos uma delas.
\begin{equation*}
  \left\{
    \begin{array}{rcl}
      -2x-6y&=& -2 \\
      -7y & =& -4
    \end{array}
  \right..
\end{equation*} Este sistema também tem as mesmas soluções do anterior, pois podemos a qualquer hora retornar ao sistema original fazendo $($linha 2$)-($linha 1$)$.

Em verdade, manter a equação da primeira linha multiplicada por $-2$ não ajuda em nada na resolução do sistema e voltaremos à equação original. Além disso, na linha dois já podemos isolar a variável $y$ e descobrir o seu valor. Ficamos então com o sistema:
\begin{equation}\label{exp1-2}
  \left\{
    \begin{array}{rcl}
      x+3y&=& 1 \\
      -7y & =& -4
    \end{array}
  \right..
\end{equation}

Observamos, finalmente, que a solução do último sistema (que é a mesma do sistema original) já está muito fácil de ser obtida. Sabemos o valor de $y$ pela segunda equação e, para descobrir o valor de $x$, basta usar a primeira equação:
\[
y = \frac{4}{7} \implies x + 3\cdot \frac{4}{7} = 1 \implies x = 1 - \frac{12}{7} = - \frac{5}{7}.
\] Podemos escrever a solução do sistema como
\begin{equation*}
  \left\{
    \begin{array}{rcl}
      x&=&-5/7 \\
      y&=& \ \ 4/7
    \end{array}
  \right. \lhd
\end{equation*}
\end{example}

O símbolo $\lhd$ acima nas nossas notas indica o fim de uma solução, exemplo ou observação.

Em seguida, fazemos algumas observações.

\begin{remark}
O sistema linear como o acima possui duas equações e duas variáveis e por isto é conhecido como um sistema linear $2\times 2$ (lê-se dois por dois). Mais geralmente, um sistema linear com $m$ equações e $n$ variáveis é conhecido como um sistema linear $m \times n$ (lê-se $m$ por $n$).
\end{remark}

\begin{remark}
Sistemas $2\times 2$ são dos mais simples de resolver e por isso o método acima pode parecer desnecessariamente complicado. Mas, como veremos, este método pode ser aplicado para qualquer sistema. Logo, é essencial que se entenda completamente todos os seus passos para um bom acompanhamento do que está por vir.
\end{remark}

Vamos analisar a solução apresentada acima mais de perto. Para chegar do sistema linear em \eqref{exp1} para o sistema em \eqref{exp1-2}, fizemos uma sequência de operações que:
\begin{itemize}
  \item \emph{não alteram a solução do sistema linear original e que;}
  \item resumidamente, podem ser descritas como \emph{adicionar um múltiplo da linha um na linha dois.}
\end{itemize} Que a solução permanece a mesma pode ser justificado tal qual está na ``Solução 1'' acima.

\vspace{0.2cm}

Operações nas linhas de um sistema que não alteram o conjunto de soluções são chamadas de \textbf{operações elementares}. São as seguintes:
\begin{enumerate}
  \item Multiplicar uma linha por uma constante;
  \item Trocar duas linhas de lugar;
  \item Somar um múltiplo de uma linha a outra linha.
\end{enumerate}

\vspace{0.2cm}

Um artifício que torna todo o processo mais ``automático'' é o de formar uma tabela de números com os coeficientes do sistema:
\begin{equation*}
  \left\{
    \begin{array}{rcl}
      x+3y&=&1 \\
      2x-y&=&-2
    \end{array}
  \right. \qquad \rightsquigarrow \qquad  \left[
                             \begin{array}{cc|c}
                               1 & 3 & 1 \\
                               2 & -1 & -2 \\
                             \end{array}
                           \right].
\end{equation*} A tabela de números da direita é conhecida como a \textbf{matriz aumentada associada ao sistema linear}. Vamos nos referir às linhas da matriz associada como $\ell_1$ e $\ell_2$. Deste modo, ``adicionar um múltiplo da linha 1 na linha dois'' corresponde a fazer:
\[
\left[
                             \begin{array}{cc|c}
                               1 & 3 & 1 \\
                               2 & -1 & -2 \\
                             \end{array}
                           \right] \quad \xrightarrow{-2\ell_1 + \ell_2 \text{ em } \ell_2} \quad
\left[
                             \begin{array}{cc|c}
                               1 & 3 & 1 \\
                               0 & -7 & -4 \\
                             \end{array}
                           \right].
\] Para este tipo de operação, olhamos para as colunas da matriz associada, fizemos mentalmente (ou em uma folha de rascunho) os cálculos envolvidos e preenchemos os valores na respectiva linha da nova matriz que está à direita:
\[
(-2)\cdot 1 + 2 = 0, \qquad (-2)\cdot 3 + (-1) = -7, \qquad (-2) \cdot 1 + (-2) = -4.
\]



\begin{example}\label{exemplo3por3}
Consideramos agora um sistema com três equações e três variáveis;
\[
\left\{
  \begin{array}{ll}
    x + 2y + z = 12 \\
    x -3y + 5z = 1 \\
    2x - y + 3z = 10
  \end{array}
\right. .
\] Vamos diretamente escrever a matriz aumentada associada ao sistema, que é a matriz com os coeficientes da variável $x$ na primeira coluna, os coeficientes da variável $y$ na segunda coluna, os coeficientes de $z$ na terceira coluna e os coeficientes independentes na última coluna:
\[
\left[
  \begin{array}{ccc|c}
    1 &  2 & 1 & 12 \\
    1 & -3 & 5 & 1 \\
    2 & -1 & 3 & 10 \\
  \end{array}
\right].
\] Em seguida, utilizamos operações elementares nas linhas como no exemplo anterior. Ao substituir $-\ell_1 + \ell_2$ em $\ell_2$, obtemos
\[
\left[
  \begin{array}{ccc|c}
    1 &  2 & 1 & 12 \\
    0 & -5 & 4 & -11 \\
    2 & -1 & 3 & 10 \\
  \end{array}
\right].
\] Ou seja, eliminamos o valor $1$ da primeira posição da segunda linha (em outras palavras, eliminamos a variável $x$ da segunda equação). Prosseguindo com a eliminação dos valores abaixo da diagonal principal da matriz, obtemos:
\[
\left[
  \begin{array}{ccc|c}
    1 &  2 & 1 & 12 \\
    0 & -5 & 4 & -11 \\
    2 & -1 & 3 & 10 \\
  \end{array}
\right]
\quad \xrightarrow{-2\ell_1 + \ell_3 \text{ em } \ell_3} \quad
\left[
  \begin{array}{ccc|c}
    1 &  2 & 1 &  12 \\
    0 & -5 & 4 & -11 \\
    0 & -5 & 1 & -14 \\
  \end{array}
\right]
\quad \xrightarrow{-\ell_2 + \ell_3 \text{ em } \ell_3} \quad
\left[
  \begin{array}{ccc|c}
    1 &  2 &  1 &  12 \\
    0 & -5 &  4 & -11 \\
    0 &  0 & -3 & -3 \\
  \end{array}
\right].
\] Voltando à notação original do sistema, obtemos
\[
\left\{
  \begin{array}{llll}
    x &+ 2y &+ z &= 12 \\
      &- 5y &+ 4z &= -11 \\
      &    &- 3z& = -3
  \end{array}
\right.
\] Por causa do seu formato, este sistema é dito estar em \textbf{forma triangular} ou ainda que o sistema é \textbf{triangular}.

Como já tínhamos percebido no nosso primeiro exemplo, sistemas triangulares são fáceis de resolver. Pela última equação concluimos que $z=1$. Subsituindo este valor na segunda equação, obtemos
\[
- 5y + 4 = -11 \implies y = 3.
\] Finalmente, substituindo estes dois valores na primeira equação, obtemos
\[
x + 2\cdot 3 + 1 = 12 \implies x=5.
\] Portanto, a solução para o sistema é $(x,y,z) = (5,3,1). \lhd$
\end{example}


Esta técnica de resolver sistemas lineares é importante por ser aplicável a sistemas de qualquer ordem e é conhecida como \textbf{escalonamento} ou \textbf{eliminação Gaussiana}.

\section{Formas escalonadas e formas escalonadas reduzidas}


Vimos dois exemplos de como resolver sistemas lineares por escalonamento. Vamos agora introduzir uma terminologia, esperando tornar o método mais sistemático.

Dizemos que uma matriz está em \textbf{forma escalonada} quando
\begin{enumerate}
  \item As linhas que contém apenas zeros estão abaixo das demais.
  \item O primeiro elemento não nulo de uma linha, conhecido como \textbf{elemento líder}, está em uma coluna à direita do elemento líder da linha acima.
\end{enumerate} Estas duas condições \emph{implicam} que os elementos que estão abaixo de elementos líder são todos iguais a zero.

Por exemplo,
\begin{equation}\label{escalonada1}
\left[
\begin{array}{ccccc}
1 & a_0 & a_1 & a_2 & a_3 \\
0 & 0 & 2 & a_4 & a_5 \\
0 & 0 & 0 & -1 & a_6 \\
0 & 0 & 0 & 0 & 0
\end{array}
\right]
\end{equation} é uma matriz em forma escalonada. Notamos que os coeficientes $a_1, a_2, a_3, a_4, a_5$ e $a_6$ podem ser quaisquer números sem alterar o fato de a matriz estar em forma escalonada.

Outros exemplos, retirados livro do David C. Lay:
\[
\left[
\begin{array}{cccc}
\blacksquare & a_0 & a_1 & a_2 \\
0 & \blacksquare & a_3 & a_4 \\
0 & 0 & 0 & 0 \\
0 & 0 & 0 & 0
\end{array}
\right], \quad
\left[
\begin{array}{cccccccccc}
0 & \blacksquare & a_1 & a_2 & a_3 & a_4 & a_5 & a_6 & a_7 & a_8\\
0 & 0 & 0 & \blacksquare & a_9 & a_{10} & a_{11} & a_{12} & a_{13} & a_{14}\\
0 & 0 & 0 & 0 & \blacksquare & a_{15} & a_{16} & a_{17} & a_{18} & a_{19}\\
0 & 0 & 0 & 0 & 0 & \blacksquare & a_{20} & a_{21} & a_{22} & a_{23} \\
0 & 0 & 0 & 0 & 0 & 0 & 0 & 0 & \blacksquare & a_{24}
\end{array}
\right].
\] Quando uma matriz está em forma escalonada, as posições marcadas com $\blacksquare$ são chamadas de \textbf{posições de pivô}. Observe que, caso a matriz não esteja em forma escalonada, o elemento líder de uma linha pode não estar na posição de pivô. Dizemos também que uma coluna é uma \textbf{coluna pivô} quando a coluna possui uma posição de pivô. Por exemplo, na primeira matriz acima, as duas primeiras colunas são colunas pivô enquanto a terceira e a quarta não são.

\vspace{0.2cm}

Uma matriz estará em \textbf{forma escalonada reduzida} quando:
\begin{enumerate}
  \item Está na forma escalonada;
  \item Todos os elementos líder são iguais a 1 e são os únicos elementos não nulos das suas colunas.
\end{enumerate} Por exemplo,
\[
\left[
\begin{array}{ccccc}
   1 & 0 & b_{1} & 0 & b_{2}\\
   0 & 1 & b_{3} & 0 & b_{3}\\
   0 & 0 & 0 & 1 & b_{4}
\end{array}
\right]
\] é uma forma escalonada reduzida da matriz da fórmula \eqref{escalonada1} acima.

Quanto aos outros exemplos, temos
\[
\left[
\begin{array}{cccc}
1 & 0 & b_1 & b_2 \\
0 & 1 & b_3 & b_4 \\
0 & 0 & 0 & 0 \\
0 & 0 & 0 & 0
\end{array}
\right], \quad
\left[
\begin{array}{cccccccccc}
0 & 1 & b_1 & 0 & 0 & 0 & b_5 & b_6 & 0 & b_8 \\
0 & 0 & 0 & 1 & 0 &0 & b_{11} & b_{12} & 0 & b_{14} \\
0 & 0 & 0 & 0 & 1 & 0 & b_{16} & b_{17} & 0 & b_{19} \\
0 & 0 & 0 & 0 & 0 & 1 & b_{20} & b_{21} & 0 & b_{23} \\
0 & 0 & 0 & 0 & 0 & 0 & 0 & 0 & 1 & b_{24}
\end{array}
\right].
\]

\begin{remark}
A forma escalonada reduzida ainda não apareceu na seção anterior quando estávamos resolvendo sistemas, mas notamos que poderia ter sido utilizada para automatizar ainda mais o processo de resolução do sistema. Vamos retomar o Exemplo \ref{exemplo3por3}. Fazendo mais algumas operações elementares, é possível transformar a matriz aumentada do sistema em uma matriz em forma escalonada reduzida:
\[
\left[
  \begin{array}{ccc|c}
    1 &  2 &  1 &  12 \\
    0 & -5 &  4 & -11 \\
    0 &  0 & -3 & -3 \\
  \end{array}
\right]
\quad \xrightarrow{\ell_2 \div (-5)  \text{ e } \ell_3\div (-3)} \quad
\left[
  \begin{array}{ccc|c}
    1 &  2 &  1   &  12 \\
    0 & 1  & -4/5 & 11/5 \\
    0 &  0 &  1   &  1 \\
  \end{array}
\right]
\]
\[
\xrightarrow{\frac{4}{5}\ell_3 + \ell_2 \text{ em } \ell_2} \quad
\left[
  \begin{array}{ccc|c}
    1 &  2 &  1   &  12 \\
    0 & 1  &  0   &  3 \\
    0 &  0 &  1   &  1 \\
  \end{array}
\right]
\xrightarrow{-\ell_3 + \ell_1 \text{ em } \ell_1} \quad
\left[
  \begin{array}{ccc|c}
    1 &  2 &  0   &  11 \\
    0 &  1 &  0   &  3 \\
    0 &  0 &  1   &  1 \\
  \end{array}
\right]
\xrightarrow{-2\ell_2 + \ell_1 \text{ em } \ell_1} \quad
\left[
  \begin{array}{ccc|c}
    1 &  0 &  0   &  5 \\
    0 &  1 &  0   &  3 \\
    0 &  0 &  1   &  1
  \end{array}
\right].
\] A última matriz acima está em forma escalonada reduzida e está associada ao sistema
\[
\left\{
  \begin{array}{llll}
    x &   &   & = 5 \\
      & y &   & = 3 \\
      &   & z & = 1
  \end{array}
\right.
\] que, na verdade, já é a solução explícita do sistema original$.\lhd$
\end{remark}

\section{Algoritmo de escalonamento}

Sistematicamente, encontramos a forma escalonada de uma matriz aplicando os seguintes passos:
\begin{enumerate}
  \item Analisar a primeira coluna pivô -- da esquerda para a direita, esta é a primeira coluna que possui algum elemento diferente de zero -- e, se necessário, aplicar operações elementares para que o elemento da primeira linha (esta é a posição de pivô!) passe a ser diferente de zero;
  \item A partir de operações elementares, eliminar todos elementos que estão abaixo do elemento na posição de pivô que obtivemos no Passo 1;
  \item Desconsiderando por um momento a primeira linha, procuramos pela próxima coluna pivô -- aquela que tem algum elemento não nulo \textit{abaixo da primeira linha}. Se necessário, aplicar operações elementares para que, na segunda linha, o elemento passe a ser diferente de zero;
  \item A partir de operações elementares, eliminar todos elementos que estão abaixo do elemento na posição de pivô que obtivemos no Passo 3;
  \item Desconsiderando por um momento a primeira e a segunda linhas, procuramos pela próxima coluna pivô -- aquela que tem algum elemento não nulo \textit{abaixo da primeira linha e da segunda linha}. Se necessário, aplicar operações elementares para que, na segunda linha, o elemento passe a ser diferente de zero;
  \item A partir de operações elementares, eliminar todos elementos que estão abaixo do elemento na posição de pivô que obtivemos no Passo 5 e ssim por diante.
\end{enumerate} Essencialmente, são apenas dois passos que se repetem até que se esgotem as colunas que possuem posição de pivô. Vejamos um exemplo.

\begin{example}\label{escalonada2}
Considere a matriz
\[
\left[
\begin{array}{rrrrrr}
   0&0&0&0&0&0 \\
   0&0&2&1&-1&8\\
   0&0&-3&-1&2&-11\\
   0&0&6&2&-4&22\\
   0&0&-2&1&2&-3
\end{array}
\right].
\]
\textit{Passo 1.} A primeira coluna pivô é a terceira. Escolhemos um elemento não nulo da terceira coluna para ocupar a posição de pivô. Por exemplo, a segunda linha. Trocando a primeira e a segunda linhas de lugar (esta é uma operação elementar), obtemos:
\[
\left[
\begin{array}{rrrrrr}
   0&0&2&1&-1&8\\
   0&0&0&0&0&0 \\
   0&0&-3&-1&2&-11\\
   0&0&6&2&-4&22\\
   0&0&-2&1&2&-3
\end{array}
\right].
\]
\textit{Passo 2.} Eliminar os elementos abaixo do 2 que está na posição de pivô da terceira coluna:
\[
\left[
\begin{array}{rrrrrr}
   0&0&2&1&-1&8\\
   0&0&0&0&0&0 \\
   0&0&-3&-1&2&-11\\
   0&0&6&2&-4&22\\
   0&0&-2&1&2&-3
\end{array}
\right]
\xrightarrow{\frac{3}{2}\ell_1 + \ell_3 \text{ em } \ell_3} \quad
\left[
\begin{array}{cccccc}
   0&0&2&1&-1&8\\
   0&0&0&0&0&0 \\
   0&0&0&1/2&1/2&1\\
   0&0&6&2&-4&22\\
   0&0&-2&1&2&-3
\end{array}
\right]
\]
\[
\xrightarrow{-3\ell_1 + \ell_4 \text{ em } \ell_4} \quad
\left[
\begin{array}{cccccc}
   0&0&2&1&-1&8\\
   0&0&0&0&0&0 \\
   0&0&0&1/2&1/2&1\\
   0&0&0&-1&-1&-2\\
   0&0&-2&1&2&-3
\end{array}
\right]
\xrightarrow{ \ell_1 + \ell_5 \text{ em } \ell_5} \quad
\left[
\begin{array}{cccccc}
   0&0&2&1&-1&8\\
   0&0&0&0&0&0 \\
   0&0&0&1/2&1/2&1\\
   0&0&0&-1&-1&-2\\
   0&0&0&2&1&5
\end{array}
\right].
\]

\textit{Passo 3.} A partir de agora, vamos essencialmente repetir o processo já realizado até agora. Notamos que a próxima coluna que contém elementos não nulos (sem olhar para a primeira linha!) é a quarta coluna. Portanto, esta é uma coluna pivô e vamos posicionar um elemento não nulo na segunda linha. Por exemplo, podemos trocar a segunda e a terceira linhas.
\[
\left[
\begin{array}{cccccc}
   0&0&2&1&-1&8\\
   0&0&0&1/2&1/2&1\\
   0&0&0&0&0&0 \\
   0&0&0&-1&-1&-2\\
   0&0&0&2&1&5
\end{array}
\right].
\] Antes de prosseguir, podemos ainda simplificar alguns cálculos multiplicando linhas por escalares (fazer isso é realizar um operação elementar!):
\[
\left[
\begin{array}{cccccc}
   0&0&2&1&-1&8\\
   0&0&0&1/2&1/2&1\\
   0&0&0&0&0&0 \\
   0&0&0&-1&-1&-2\\
   0&0&0&2&1&5
\end{array}
\right]
\xrightarrow{ 2\cdot\ell_3 \text{ e } (-1)\cdot \ell_4} \quad
\left[
\begin{array}{cccccc}
   0&0&2&1&-1&8\\
   0&0&0&1&1&2\\
   0&0&0&0&0&0\\
   0&0&0&1&1&2\\
   0&0&0&2&1&5
\end{array}
\right]
\]
\textit{Passo 4.} Prosseguimos como no Passo 2 para eliminar todos os elementos que estão abaixo da posição de pivô da quarta coluna:
\[
\left[
\begin{array}{cccccc}
   0&0&2&1&-1&8\\
   0&0&0&1&1&2\\
   0&0&0&0&0&0\\
   0&0&0&1&1&2\\
   0&0&0&2&1&5
\end{array}
\right]
\xrightarrow{- \ell_2 + \ell_4 \text{ em } \ell_4} \quad
\left[
\begin{array}{cccccc}
   0&0&2&1&-1&8\\
   0&0&0&1&1&2\\
   0&0&0&0&0&0\\
   0&0&0&0&0&0\\
   0&0&0&2&1&5
\end{array}
\right]
\]
\[
\xrightarrow{- 2\ell_2 + \ell_5 \text{ em } \ell_5} \quad
\left[
\begin{array}{cccccc}
   0&0&2&1&-1&8\\
   0&0&0&1&1&2\\
   0&0&0&0&0&0\\
   0&0&0&0&0&0\\
   0&0&0&0&-1&1
\end{array}
\right]
\]
\textit{Passo 5.} Finalmente, identificamos a coluna 5 como coluna pivô e obtemos uma matriz em forma escalonada ao deixar um elemento não nulo na posição de pivô:
\[
\left[
\begin{array}{cccccc}
   0&0&2&1&-1&8\\
   0&0&0&1&1&2\\
   0&0&0&0&-1&1\\
   0&0&0&0&0&0\\
   0&0&0&0&0&0
\end{array}
\right]. \lhd
\]
\end{example}

Agora, para obter a forma escalonada reduzida \emph{a partir da forma escalonada}, primeiramente aplicamos os passos acima para obter uma matriz em forma escalonada. Em seguida, aplicamos outros passos iterativos:
\begin{enumerate}
  \item Começar pela posição de pivô mais à \emph{direita} e eliminar os elementos não nulos \emph{acima} da posição de pivô;
  \item Se necessário, dividir a linha pelo valor do elemento líder (que está na posição de pivô) para que o elemento líder fique igual a 1;
  \item Repetir os primeiros passos para o elemento líder na próxima (da direita para a esquerda) coluna pivô.
\end{enumerate}

Observamos que poderíamos ter primeiro realizado o Passo 2 e depois o Passo 1 se julgássemos que seria mais simples para efetuar os cálculos.

\begin{example}
    Voltamos ao Exemplo \ref{escalonada2}. Identificando novamente as posições de pivô:
\[
\left[
\begin{array}{cccccc}
   0&0&\mathbf{2}&1&-1&8\\
   0&0&0&\mathbf{1}&1&2\\
   0&0&0&0&\mathbf{-1}&1\\
   0&0&0&0&0&0\\
   0&0&0&0&0&0
\end{array}
\right].
\] O mais da direita é o $-1$ da quinta coluna. Eliminando os termos não nulos acima, obtemos:
\[
\left[
\begin{array}{cccccc}
   0&0&2&1&-1&8\\
   0&0&0&1&1&2\\
   0&0&0&0&-1&1\\
   0&0&0&0&0&0\\
   0&0&0&0&0&0
\end{array}
\right]
\xrightarrow{\ell_3 + \ell_2 \text{ em } \ell_2} \quad
\left[
\begin{array}{cccccc}
   0&0&2&1&-1&8\\
   0&0&0&1&0&3\\
   0&0&0&0&-1&1\\
   0&0&0&0&0&0\\
   0&0&0&0&0&0
\end{array}
\right]
\xrightarrow{-\ell_3 + \ell_1 \text{ em } \ell_1} \quad
\left[
\begin{array}{cccccc}
   0&0&2&1&0&7\\
   0&0&0&1&0&3\\
   0&0&0&0&-1&1\\
   0&0&0&0&0&0\\
   0&0&0&0&0&0
\end{array}
\right].
\] A próxima posição de pivô mais à direita é o 1 na linha 2, coluna 4. Eliminando o termo não nulo acima, obtemos:
\[
\xrightarrow{-\ell_2 + \ell_1 \text{ em } \ell_1} \quad
\left[
\begin{array}{cccccc}
   0&0&2&0&0&4\\
   0&0&0&1&0&3\\
   0&0&0&0&-1&1\\
   0&0&0&0&0&0\\
   0&0&0&0&0&0
\end{array}
\right].
\] Finalmente, dividimos a primeira linha por $2$ e a terceira linha por $-1$ para chegar na forma escalonada reduzida da matriz inicial:
\[
\left[
\begin{array}{cccccc}
   0&0&1&0&0&2\\
   0&0&0&1&0&3\\
   0&0&0&0&1&-1\\
   0&0&0&0&0&0\\
   0&0&0&0&0&0
\end{array}
\right].
\]
\end{example}


\section{Existência e unicidade de solução}


\subsection{Sistemas que possuem apenas uma solução}

Os sistemas que vimos até agora possuiam apenas uma solução. Esta propriedade, como veremos, nem sempre é válida. No entanto, é fácil de identificar quando um sistema possui solução única analisando a forma escalonada da matriz associada: \textbf{quando todas as colunas referentes às variáveis da equação possuirem posição de pivô}. Por exemplo,
\[
\left[
 \begin{array}{cc|c}
  1 & 3 & 1 \\
  0 & -7 & -4 \\
 \end{array}
\right],
\left[
  \begin{array}{ccc|c}
    1 &  2 &  1 &  12 \\
    0 & -5 &  4 & -11 \\
    0 &  0 & -3 & -3
  \end{array}
\right],
 \left[
  \begin{array}{cccc|c}
    2 &  0 &  0 & 0 & 10 \\
    0 &  1 &  0 & 0 & -12 \\
    0 &  0 &  2 & 0 & 6 \\
    0 &  0 &  0 & 1 & 1
  \end{array}
\right].
\] De fato, todas as matrizes com esta propriedade tem em sua forma escalonada reduzida a matriz identidade do lado esquerdo da barra que separa as variáveis e os valores do lado direito da igualdade do sistema:
\[
\left[
 \begin{array}{cc|c}
  1 & 0 & -5/7 \\
  0 & 1 &  4/7 \\
 \end{array}
\right],
\left[
  \begin{array}{ccc|c}
    1 &  0 &  0   &  5 \\
    0 &  1 &  0   &  3 \\
    0 &  0 &  1   &  1
  \end{array}
\right],
 \left[
  \begin{array}{cccc|c}
    1 &  0 &  0 & 0 & 5 \\
    0 &  1 &  0 & 0 & -12 \\
    0 &  0 &  1 & 0 & 3 \\
    0 &  0 &  0 & 1 & 1
  \end{array}
\right].
\] Escrevendo o sistema associado as estas últimas matrizes, é fácil de ver que a solução será única.


\subsection{Sistemas impossíveis ou inconsistentes}

Nem todo sistema linear possui solução. Um exemplo simples deste fato é
\begin{equation}\label{sistemaimpossivel}
\left\{
  \begin{array}{ll}
    x+y = 1\\
    x+y = 2
  \end{array}
\right. .
\end{equation} É claro que $x+y$ não pode ser ao mesmo tempo igual a 1 e igual a 2; portanto, o sistema não possui solução. O fato de o sistema não possuir solução -- ser \textbf{inconsistente} -- nem sempre é tão fácil de ser identificado apenas olhando para o sistema. Mas esta propriedade salta aos olhos quando analisamos a forma escalonada da matriz associada ao sistema. De fato, na matriz escalonada de um sistema inconsistente aparece uma linha da forma
\[
\left[
\begin{array}{ccccc|c}
   0&0&0&0&0&a
\end{array}
\right],
\] com uma constante $a \neq 0$. Esta linha representa no sistema uma equação do tipo $0 = a \neq 0$, que é impossível de ser satisfeita.

No sistema da fórmula \eqref{sistemaimpossivel} acima, temos a matriz aumentada
\[
\left[
\begin{array}{cc|c}
   1 & 1 & 1 \\
   1 & 1 & 2 \\
\end{array}
\right]
\xrightarrow{-\ell_1 + \ell_2 \text{ em } \ell_2} \quad
\left[
\begin{array}{cc|c}
   1 & 1 & 1 \\
   0 & 0 & 1 \\
\end{array}
\right].
\] Logo, nosso sistema é equivalente ao sistema impossível
\[
\left\{
  \begin{array}{rl}
    x+y &= 1\\
     0  &= 1
  \end{array}
\right. .
\]


\subsection{Sistemas com mais de uma solução}

Com exceção da subseção anterior, todos os sistemas que resolvemos anteriormente possuiam apenas uma solução. Vejamos um exemplo onde este não é o caso.

\begin{example}
Resolver
\[
\left\{
  \begin{array}{rrrrrrl}
     & 3 x_2 & -6x_3 & + 6x_4 & + 4x_5 &=& -5\\
    3x_1 & -7x_2 & +8x_3 & -5 x_4 & + 8x_5 &=& 9\\
    3x_1 & -9x_2 & +12x_3 & -9x_4 & +6x_5 &=& 15
  \end{array}
\right. .
\] A matriz aumentada associada a este sistema é
\[
\left[
  \begin{array}{ccccc|c}
    0 &  3 & -6 &  6 & 4 & -5 \\
    3 & -7 &  8 & -5 & 8 &  9 \\
    3 & -9 & 12 & -9 & 6 & 15
  \end{array}
\right] .
\] Transformando esta matriz em sua forma escalonada reduzida (faça todos os cálculos!), obtemos
\[
\left[
  \begin{array}{ccccc|c}
    1 & 0 & -2 & 3 & 0 & -24 \\
    0 & 1 & -2 & 2 & 0 &  -7 \\
    0 & 0 &  0 & 0 & 1 &   4
  \end{array}
\right],
\] que é equivalente ao sistema linear
\[
\left\{
  \begin{array}{rcl}
      x_1 - 2 x_3 + 3 x_4 &=& -24\\
      x_2 - 2 x_3 + 2 x_4  &=& -7\\
      x_5 &=& 4
  \end{array}
\right.
\] Neste sistema $x_5$ possui um valor fixo igual a 4, mas as outras variáveis tem dois graus de liberdade. De fato, atribuindo valores quaisquer para $x_3$ e $x_4$, obtemos os valores de $x_1$ e $x_2$ pelas duas primeiras equações acima e temos uma solução do sistema. Diferentes escolhas dos parâmetros $x_3$ e $x_4$ geram soluções diferentes.

Uma forma sistemática de analisar a situação é a seguinte: As variáveis correspondentes a colunas que possuem posição de pivô são chamadas de \textbf{variáveis dependentes} e as demais são chamadas de \textbf{variáveis livres}. Assim, podemos escrever
\[
\left\{
  \begin{array}{l}
       x_1  = 2 x_3 - 3 x_4 -24 \\
       x_2 = 2 x_3 - 2 x_4  -7  \\
       x_3, x_4 \text{ livres } \\
       x_5  = 4
  \end{array}
\right.
\] e o sistema possui infinitas soluções.$\lhd$
\end{example}


\subsection{Conclusões}

Ao analisar a matriz aumentada associada a um sistema linear, concluiremos \textit{inevitavelmente} que uma das seguintes situações é válida:
\begin{itemize}
  \item Todas as colunas referentes às variáveis do sistema possuem posição de pivô, \emph{e.g},
\[
\left[
  \begin{array}{ccc|c}
    1 &  3 & -6 &  6  \\
    0 & -8 & 11 & \pi  \\
    0 & 0 & 12 & -9
  \end{array}
\right].
\] Neste caso, vimos que o sistema possui \textbf{apenas uma solução}. Nota que \textit{esta possibilidade apenas é possível se tivermos um sistema com o número de equações igual ao número de variáveis}.
  \item Caso a coluna referentes à alguma variável do sistema não possua posição de pivô, teremos duas possibilidades
\begin{enumerate}[$(i)$]
  \item Alguma coluna não é coluna pivô mas não existem linhas inconsistentes, \emph{e.g},
\[
\left[
  \begin{array}{ccccc|c}
    1 & 3  & 2  & -6 & 2  &  6  \\
    0 & -8 & 4 & 11  & -1  & -3  \\
    0 & 0  &  0 & 12 &  1 & -9
  \end{array}
\right].
\] Neste caso, as variáveis referentes a colunas que não são pivô (na matriz acima, colunas 3 e 5) podem ser consideradas variáveis livres e o sistema \textbf{possui infinitas soluções}.

  \item Alguma coluna não é coluna pivô mas existem linhas que são inconsistentes, \emph{e.g},
\[
\left[
  \begin{array}{ccccc|c}
    1 & 3  & 2  & -6 & 2  &  6  \\
    0 & -8 & 4 & 11  & -1  & 1  \\
    0 & 0  &  0 & 0 &  0 & -9
  \end{array}
\right].
\] Neste caso, \textbf{não existem soluções} para o sistema, pois uma das equações é impossível de ser satisfeita.
\end{enumerate}
\end{itemize}


Concluimos assim que \textit{um sistema linear ou não possui solução, ou tem apenas uma solução ou então possui infinitas soluções}. São estes três todos os casos possíveis e podemos decidir em qual estamos apenas olhando para a forma escalonada da matriz aumentada associada ao sistema.


\end{document} 