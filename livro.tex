%Este trabalho est� licenciado sob a Licen�a Creative Commons Atribui��o-CompartilhaIgual 3.0 N�o Adaptada. Para ver uma c�pia desta licen�a, visite http://creativecommons.org/licenses/by-sa/3.0/ ou envie uma carta para Creative Commons, PO Box 1866, Mountain View, CA 94042, USA.


%%%%%%%%%%%%%%%%%%%%%%%%%%%%%%%%%
%
% ATEN��O: N�O EDITE ESTE ARQUIVO
%
%%%%%%%%%%%%%%%%%%%%%%%%%%%%%%%%%


\documentclass[a4paper]{book}

%%%%%%%%%%%%%%%%%%%%%%%%%%%%%%%%%%%%%%%%%%%%%%
%%%%%%%%Pacotes b�sicos para MathEnvir%%%%%%%%
%%%%%%%%%%%%%%%%%%%%%%%%%%%%%%%%%%%%%%%%%%%%%%


\usepackage{amsmath}%%AMS primary package (includes amstext, amsopn, amsbsy), provides various features for displayed equations and %%other mathematical constructs.

%%%%OPTIONS FOR THE AMSMATH PACKAGE
\usepackage{amscd}     %%Provides a CD environment for simple commutative diagrams (no support for diagonal arrows).
\usepackage{amsxtra}   %%Provides certain odds and ends such as \fracwithdelims and \accentedsymbol, for compatibility with documents %%created using version 1.1.

\usepackage{amsthm}    %%Enhanced version of \newtheorem command for defining theorem-like environments

\usepackage{amssymb}   %%Provides an extended symbol collection (includes amsfonts). For example, \barwedge, \boxdot, \boxminus, %%\boxplus, \boxtimes, \Cap, \Cup (and many more), the arrow \leadsto, and some other symbols such as \Box and \Diamond.

\usepackage{latexsym}  %%makes few additional characters available: \Box \Join \Box \Diamond \leadsto \sqsubset \sqsupset \lhd \unlhd %%\rhd \unrhd

\usepackage[makeroom]{cancel}   % Cancelar termos em equa��es

\usepackage{enumerate}

%%%%%%%%%%%%%%%%%%%%%%%%%%%%%%%%%%%%%%%%%%%%%%
%%%%%%Color, graphicx, margin (geometry)%%%%%%
%%%%%%%%%%%%%%%%%%%%%%%%%%%%%%%%%%%%%%%%%%%%%%


\usepackage[dvips,pdftex]{graphicx}
%\usepackage{color}            %%Textcolor, color definitions, etc

%\definecolor{light-blue}{rgb}{0.8,0.85,1}     %%Numbers between 0 and 1
%\definecolor{mygrey}{gray}{0.75}              %%Numbers between 0 and 1

\usepackage{verbatim}         %%Adds text from other files, comment environment

\usepackage{xpatch}           %%Bold theorem titles
\makeatletter
   \xpatchcmd{\@thm}{\fontseries\mddefault\upshape}{}{}{} %same font as thm-header
\makeatother

\usepackage[margin=1in]{geometry}  %%Margins  %%Possible to use \newgeometry to modify small parts mid-document


%font
\usepackage{tgbonum}



%%%%%%%%%%%%%%%%%%%%%%%%%%%%%%%%%%%%%%%%%%%%%
%%%%%%%%%%%Para escrever portugu�s%%%%%%%%%%%
%%%%%%%%%%%%%%%%%%%%%%%%%%%%%%%%%%%%%%%%%%%%%

\usepackage[portuguese]{babel}   %%Portuguese-specific commands


%%%%%%%%%%%%%%%%%%%%%%%%%%%%%%%%%%%%%%%%%%%%%
%%%%%%%%%%Theorem styles, numbering%%%%%%%%%%
%%%%%%%%%%%%%%%%%%%%%%%%%%%%%%%%%%%%%%%%%%%%%


\theoremstyle{plain}
\newtheorem{theorem}{Teorema}              %%section, chapter, etc
\newtheorem{proposition}[theorem]{Proposi��o}      %%everything below obeys theorem because: [theorem]
\newtheorem{lemma}[theorem]{Lema}
\newtheorem{corollary}[theorem]{Corol�rio}
\newtheorem{maintheorem}{Teorema}                   %%number doesn't obey order
\renewcommand{\themaintheorem}{\Alph{maintheorem}}
\newtheorem{maincorollary}{Corol�rio}
\newtheorem{conjecture}{Conjectura}
\newtheorem*{claim}{Afirma��o}

\newtheorem*{desloct}{Segundo Teorema de Deslocamento -- Deslocamento em $t$}     %%Sem numera��o e com o nome desejado

\theoremstyle{definition}
\newtheorem{remark}[theorem]{Observa��o}
\newtheorem{example}[theorem]{Exemplo}
\newtheorem{definition}[theorem]{Defini��o}
\newtheorem{exercise}{Exerc�cio}


%%%%%%%%%%%%%%%%%%%%%%%%%%%%%%%%%%%%%%%%%%%%%
%%%%%%%%%%%%%%%%New Commands%%%%%%%%%%%%%%%%%
%%%%%%%%%%%%%%%%%%%%%%%%%%%%%%%%%%%%%%%%%%%%%

\newcommand{\mb}[1]{\mathbb{#1}}
\newcommand{\bC}{\mathbb{C}}
\newcommand{\bE}{\mathbb{E}}
\newcommand{\bK}{\mathbb{K}}
\newcommand{\bN}{\mathbb{N}}
\newcommand{\bP}{\mathbb{P}}
\newcommand{\bQ}{\mathbb{Q}}
\newcommand{\bR}{\mathbb{R}}
\newcommand{\bS}{\mathbb{S}}
\newcommand{\bT}{\mathbb{T}}
\newcommand{\bZ}{\mathbb{Z}}

\newcommand{\cE}{\mathcal{E}}
\newcommand{\cF}{\mathcal{F}}
\newcommand{\cH}{\mathcal{H}}
\newcommand{\cL}{\mathcal{L}}
\newcommand{\cM}{\mathcal{M}}
\newcommand{\cO}{\mathcal{O}}
\newcommand{\cP}{\mathcal{P}}
\newcommand{\cQ}{\mathcal{Q}}
\newcommand{\cR}{\mathcal{R}}
\newcommand{\cS}{\mathcal{S}}

\newcommand{\al} {\alpha}       \newcommand{\Al}{\Alpha}
\newcommand{\be} {\beta}        \newcommand{\Be}{\Beta}
\newcommand{\ga} {\gamma}       \newcommand{\Ga}{\Gamma}
\newcommand{\de} {\delta}       \newcommand{\De}{\Delta}
\newcommand{\ep} {\epsilon}
\newcommand{\eps}{\varepsilon}
\newcommand{\ze} {\zeta}
\newcommand{\vte}{\vartheta}
\newcommand{\iot}{\iota}
\newcommand{\ka} {\kappa}
\newcommand{\la} {\lambda}      \newcommand{\La}{\Lambda}
\newcommand{\vpi}{\varpi}
%\newcommand{\ro} {\rho}
\newcommand{\vro}{\varrho}
\newcommand{\si} {\sigma}       \newcommand{\Si}{\Sigma}
\newcommand{\vsi}{\varsigma}
\newcommand{\ups}{\upsilon}     \newcommand{\Up}{\Upsilon}
\newcommand{\vphi}{\varphi}
\newcommand{\om} {\omega}       \newcommand{\Om}{\Omega}

\newcommand{\ang}{\operatorname{angle}}
\newcommand{\closu}{\operatorname{clos}}
\newcommand{\Col}{\operatorname{Col}}
\newcommand{\const}{\operatorname{const}}
\newcommand{\curl}{\operatorname{curl}}
\newcommand{\dd}{\, \mathrm{d}}
\newcommand{\diam}{\operatorname{diam}}
\newcommand{\Div}{\operatorname{div}}
\newcommand{\dist}{\operatorname{dist}}
\newcommand{\grad}{\operatorname{grad}}
\newcommand{\fr}{\partial}
\newcommand{\graph}{\operatorname{graph}}
\newcommand{\id}{\operatorname{Id}}
\newcommand{\inter}{\operatorname{int}}
\newcommand{\Leb}{\operatorname{Leb}}
\newcommand{\length}{\operatorname{length}}
\newcommand{\Lip}{\operatorname{Lip}}
\newcommand{\proj}{\operatorname{proj}}
\newcommand{\res}{\operatornamewithlimits{Res}}
\newcommand{\rot}{\operatorname{rot}}
\newcommand{\sen}{\operatorname{sen}}
\newcommand{\senh}{\operatorname{senh}}
\newcommand{\Span}{\operatorname{Span}}
\newcommand{\spec}{\operatorname{spec}}
\newcommand{\supp}{\operatorname{supp}}
\newcommand{\var}{\operatornamewithlimits{{var}}}

\everymath{\displaystyle}

%liga��es
\usepackage[pdfborder={0 0 0 [0 0]},colorlinks=true,linkcolor=blue,citecolor=blue,filecolor=blue,urlcolor=blue]{hyperref}


%define a marca��o dos cap�tulos
\addto\captionsportuguese{\renewcommand{\chaptername}{}}
%\addto\captionsportuguese{\renewcommand{\chaptermark}{}}

%%%% no blank pages between chapters %%%%
\let\cleardoublepage\clearpage


%pacote de modula��o
\usepackage{subfiles}

%cabe�alho e rodap�
\usepackage{fancyhdr}
\pagestyle{fancy}
\fancyhead{}
\fancyfoot{}
\fancyhead[L]{�lgebra Linear - Um Livro Colaborativo}
\fancyhead[R]{\thepage}
\fancyfoot[C]{\tiny{Licen�a CC-BY-SA-3.0. Contato: \url{livroscolaborativos@gmail.com}}}

\newcommand{\emconstrucao}{
  \begin{center}
    Em constru��o ... Caso deseja colaborar com a escrita deste livro, veja como em:

    \url{https://github.com/livroscolaborativos/AlgebraLinear}
  \end{center}
}

\begin{document}

%diret�rio de origem
\newcommand{\dir}{.}

\frontmatter

\title{�lgebra Linear\\\small{Um Livro Colaborativo}}

\author{}

\date{\today}

\maketitle

%Este trabalho está licenciado sob a Licença Creative Commons Atribuição-CompartilhaIgual 3.0 Não Adaptada. Para ver uma cópia desta licença, visite https://creativecommons.org/licenses/by-sa/3.0/ ou envie uma carta para Creative Commons, PO Box 1866, Mountain View, CA 94042, USA.

\chapter*{Organizadores}
\addcontentsline{toc}{chapter}{Organizadores}

\begin{itemize}
\item[] Diego Marcon Farias - UFRGS
\item[] Pedro Henrique de Almeida Konzen - UFRGS
\item[] Rafael Rigão Souza - UFRGS
\end{itemize}

%Este trabalho está licenciado sob a Licença Creative Commons Atribuição-CompartilhaIgual 3.0 Não Adaptada. Para ver uma cópia desta licença, visite http://creativecommons.org/licenses/by-sa/3.0/ ou envie uma carta para Creative Commons, PO Box 1866, Mountain View, CA 94042, USA.

\chapter*{Licença}
\addcontentsline{toc}{chapter}{Licença}

Este trabalho está licenciado sob a Licença Creative Commons Atribuição-CompartilhaIgual 3.0 Não Adaptada. Para ver uma cópia desta licença, visite http://creativecommons.org/licenses/by-sa/3.0/ ou envie uma carta para Creative Commons, PO Box 1866, Mountain View, CA 94042, USA.

%Este trabalho está licenciado sob a Licença Creative Commons Atribuição-CompartilhaIgual 3.0 Não Adaptada. Para ver uma cópia desta licença, visite http://creativecommons.org/licenses/by-sa/3.0/ ou envie uma carta para Creative Commons, PO Box 1866, Mountain View, CA 94042, USA.

\chapter*{Nota dos organizadores}
\addcontentsline{toc}{chapter}{Nota dos organizadores}

Nosso objetivo é de fomentar o desenvolvimento de materiais didáticos pela colaboração entre professores e alunos de universidades, institutos de educação e demais interessados no estudo e aplicação da álgebra linear nos mais diversos ramos da ciência e tecnologia.

Para tanto, disponibilizamos em repositório público GitHub (\url{https://github.com/reamat/Calculo}) todo o código-fonte do material em desenvolvimento sob licença Creative Commons Atribuição-CompartilhaIgual 3.0 Não Adaptada (\href{https://creativecommons.org/licenses/by-sa/3.0/}{CC-BY-SA-3.0}). Ou seja, você pode copiar, redistribuir, alterar e construir um novo material para qualquer uso, inclusive comercial. Leia a licença para maiores informações.

O sucesso do projeto depende da colaboração! Participe diretamenta da escrita dos recursos educacionais, dê sugestões ou nos avise de erros e imprecisões. Toda a colaboração é bem vinda. Veja mais sobre o projeto em:
\begin{center}
  \url{https://www.ufrgs.br/reamat/AlgebraLinear}
\end{center}

\vspace{0.5cm}

Desejamos-lhe ótimas colaborações!

%Este trabalho está licenciado sob a Licença Creative Commons Atribuição-CompartilhaIgual 3.0 Não Adaptada. Para ver uma cópia desta licença, visite http://creativecommons.org/licenses/by-sa/3.0/ ou envie uma carta para Creative Commons, PO Box 1866, Mountain View, CA 94042, USA.

\chapter*{Prefácio}
\addcontentsline{toc}{chapter}{Prefácio}

\emconstrucao


\tableofcontents
\addcontentsline{toc}{chapter}{Sum�rio}

\mainmatter

%Inclui os cap�tulos
\subfile{Semana01/semana01.tex}
\subfile{Semana02/semana02.tex}
\subfile{Semana03/semana03.tex}
\subfile{Semana04/semana04.tex}

\subfile{Semana12/semana12.tex}
\subfile{Semana13/semana13.tex}

\end{document}