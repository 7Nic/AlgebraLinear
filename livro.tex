<<<<<<< HEAD
%Este trabalho está licenciado sob a Licença Creative Commons Atribuição-CompartilhaIgual 3.0 Não Adaptada. Para ver uma cópia desta licença, visite http://creativecommons.org/licenses/by-sa/3.0/ ou envie uma carta para Creative Commons, PO Box 1866, Mountain View, CA 94042, USA.


%%%%%%%%%%%%%%%%%%%%%%%%%%%%%%%%%
%
% ATENÇÃO: NÃO EDITE ESTE ARQUIVO
%
%%%%%%%%%%%%%%%%%%%%%%%%%%%%%%%%%


\documentclass[a4paper]{book}

%%%%%%%%%%%%%%%%%%%%%%%%%%%%%%%%%%%%%%%%%%%%%%
%%%%%%%%Definições para compilação%%%%%%%%%%%%
%%%%%%%%%%%%%%%%%%%%%%%%%%%%%%%%%%%%%%%%%%%%%%
\newif\ifispdf        % O layout será pdf?
\newif\ifishtml       % O layout será html?

\def\tfn{config.knd}   % Arquivo que guarda as definições do tipo de saída
\def \tdata{}          % Definições do tipo de saída: book, slide ou html.

\openin1=\tfn\relax    % Leitura das definições de saída
\read1 to \tdata
\closein1

\tdata                 % Definições de saída

%%%%%%%%%%%%%%%%%%%%%%%%%%%%%%%%%%%%%%%%%%%%%
%%%%%%%%%%%Para escrever português%%%%%%%%%%%
%%%%%%%%%%%%%%%%%%%%%%%%%%%%%%%%%%%%%%%%%%%%%

\usepackage[portuguese]{babel}   %%Portuguese-specific commands
\usepackage[T1]{fontenc}
\usepackage[utf8]{inputenc}



%%%%%%%%%%%%%%%%%%%%%%%%%%%%%%%%%%%%%%%%%%%%%%
%%%%%%%%Pacotes básicos para MathEnvir%%%%%%%%
%%%%%%%%%%%%%%%%%%%%%%%%%%%%%%%%%%%%%%%%%%%%%%

\usepackage{amsmath}%%AMS primary package (includes amstext, amsopn, amsbsy), provides various features for displayed equations and %%other mathematical constructs.

%%%%OPTIONS FOR THE AMSMATH PACKAGE
\usepackage{amscd}     %%Provides a CD environment for simple commutative diagrams (no support for diagonal arrows).
\usepackage{amsxtra}   %%Provides certain odds and ends such as \fracwithdelims and \accentedsymbol, for compatibility with documents %%created using version 1.1.

\usepackage{amsthm}    %%Enhanced version of \newtheorem command for defining theorem-like environments

\usepackage{amssymb}   %%Provides an extended symbol collection (includes amsfonts). For example, \barwedge, \boxdot, \boxminus, %%\boxplus, \boxtimes, \Cap, \Cup (and many more), the arrow \leadsto, and some other symbols such as \Box and \Diamond.

\usepackage{latexsym}  %%makes few additional characters available: \Box \Join \Box \Diamond \leadsto \sqsubset \sqsupset \lhd \unlhd %%\rhd \unrhd

\usepackage[makeroom]{cancel}   % Cancelar termos em equações

\usepackage{enumerate}

%%%%%%%%%%%%%%%%%%%%%%%%%%%%%%%%%%%%%%%%%%%%%%
%%%%%%Color, graphicx, margin (geometry)%%%%%%
%%%%%%%%%%%%%%%%%%%%%%%%%%%%%%%%%%%%%%%%%%%%%%


\ifispdf
\usepackage[dvips]{graphicx}
\fi
\ifishtml
\usepackage[dvips]{graphicx}
\fi

%\usepackage{color}            %%Textcolor, color definitions, etc

%\definecolor{light-blue}{rgb}{0.8,0.85,1}     %%Numbers between 0 and 1
%\definecolor{mygrey}{gray}{0.75}              %%Numbers between 0 and 1

\usepackage{verbatim}         %%Adds text from other files, comment environment

\usepackage{xpatch}           %%Bold theorem titles
\makeatletter
   \xpatchcmd{\@thm}{\fontseries\mddefault\upshape}{}{}{} %same font as thm-header
\makeatother

\usepackage[margin=1in]{geometry}  %%Margins  %%Possible to use \newgeometry to modify small parts mid-document


%font
\usepackage{tgbonum}



%%%%%%%%%%%%%%%%%%%%%%%%%%%%%%%%%%%%%%%%%%%%%
%%%%%%%%%%Theorem styles, numbering%%%%%%%%%%
%%%%%%%%%%%%%%%%%%%%%%%%%%%%%%%%%%%%%%%%%%%%%


\theoremstyle{plain}
\newtheorem{theorem}{Teorema}              %%section, chapter, etc
\newtheorem{proposition}[theorem]{Proposição}      %%everything below obeys theorem because: [theorem]
\newtheorem{lemma}[theorem]{Lema}
\newtheorem{corollary}[theorem]{Corolário}
\newtheorem{maintheorem}{Teorema}                   %%number doesn't obey order
\renewcommand{\themaintheorem}{\Alph{maintheorem}}
\newtheorem{maincorollary}{Corolário}
\newtheorem{conjecture}{Conjectura}
\newtheorem*{claim}{Afirmação}

\newtheorem*{teoespectral}{Teorema Espectral}     %%Sem numeração e com o nome desejado

\theoremstyle{definition}
\newtheorem{remark}[theorem]{Observação}
\newtheorem{example}[theorem]{Exemplo}
\newtheorem{definition}[theorem]{Definição}
\newtheorem{exercise}{Exercício}


%%%%%%%%%%%%%%%%%%%%%%%%%%%%%%%%%%%%%%%%%%%%%
%%%%%%%%%%%%%%%%New Commands%%%%%%%%%%%%%%%%%
%%%%%%%%%%%%%%%%%%%%%%%%%%%%%%%%%%%%%%%%%%%%%

\newcommand{\mb}[1]{\mathbb{#1}}
\newcommand{\bC}{\mathbb{C}}
\newcommand{\bE}{\mathbb{E}}
\newcommand{\bK}{\mathbb{K}}
\newcommand{\bN}{\mathbb{N}}
\newcommand{\bP}{\mathbb{P}}
\newcommand{\bQ}{\mathbb{Q}}
\newcommand{\bR}{\mathbb{R}}
\newcommand{\bS}{\mathbb{S}}
\newcommand{\bT}{\mathbb{T}}
\newcommand{\bZ}{\mathbb{Z}}

\newcommand{\cB}{\mathcal{B}}
\newcommand{\cE}{\mathcal{E}}
\newcommand{\cF}{\mathcal{F}}
\newcommand{\cH}{\mathcal{H}}
\newcommand{\cL}{\mathcal{L}}
\newcommand{\cM}{\mathcal{M}}
\newcommand{\cO}{\mathcal{O}}
\newcommand{\cP}{\mathcal{P}}
\newcommand{\cQ}{\mathcal{Q}}
\newcommand{\cR}{\mathcal{R}}
\newcommand{\cS}{\mathcal{S}}

\newcommand{\al} {\alpha}       \newcommand{\Al}{\Alpha}
\newcommand{\be} {\beta}        \newcommand{\Be}{\Beta}
\newcommand{\ga} {\gamma}       \newcommand{\Ga}{\Gamma}
\newcommand{\de} {\delta}       \newcommand{\De}{\Delta}
\newcommand{\ep} {\epsilon}
\newcommand{\eps}{\varepsilon}
\newcommand{\ze} {\zeta}
\newcommand{\vte}{\vartheta}
\newcommand{\iot}{\iota}
\newcommand{\ka} {\kappa}
\newcommand{\la} {\lambda}      \newcommand{\La}{\Lambda}
\newcommand{\vpi}{\varpi}
%\newcommand{\ro} {\rho}
\newcommand{\vro}{\varrho}
\newcommand{\si} {\sigma}       \newcommand{\Si}{\Sigma}
\newcommand{\vsi}{\varsigma}
\newcommand{\ups}{\upsilon}     \newcommand{\Up}{\Upsilon}
\newcommand{\vphi}{\varphi}
\newcommand{\om} {\omega}       \newcommand{\Om}{\Omega}

\newcommand{\ang}{\operatorname{angle}}
\newcommand{\closu}{\operatorname{clos}}
\newcommand{\Col}{\operatorname{Col}}
\newcommand{\const}{\operatorname{const}}
\newcommand{\curl}{\operatorname{curl}}
\newcommand{\dd}{\, \mathrm{d}}
\newcommand{\diam}{\operatorname{diam}}
\newcommand{\Dim}{\operatorname{dim}}
\newcommand{\Div}{\operatorname{div}}
\newcommand{\dist}{\operatorname{dist}}
\newcommand{\grad}{\operatorname{grad}}
\newcommand{\fr}{\partial}
\newcommand{\graph}{\operatorname{graph}}
\newcommand{\id}{\operatorname{Id}}
\newcommand{\inter}{\operatorname{int}}
\newcommand{\Leb}{\operatorname{Leb}}
\newcommand{\length}{\operatorname{length}}
\newcommand{\Lip}{\operatorname{Lip}}
\newcommand{\Nul}{\operatorname{Nul}}
\newcommand{\proj}{\operatorname{proj}}
\newcommand{\res}{\operatornamewithlimits{Res}}
\newcommand{\rot}{\operatorname{rot}}
\newcommand{\sen}{\operatorname{sen}}
\newcommand{\senh}{\operatorname{senh}}
\newcommand{\Span}{\operatorname{Span}}
\newcommand{\spec}{\operatorname{spec}}
\newcommand{\supp}{\operatorname{supp}}
\newcommand{\var}{\operatornamewithlimits{{var}}}

\everymath{\displaystyle}

%ligações
\usepackage[pdfborder={0 0 0 [0 0]},colorlinks=true,linkcolor=blue,citecolor=blue,filecolor=blue,urlcolor=blue]{hyperref}


%define a marcação dos capítulos
\addto\captionsportuguese{\renewcommand{\chaptername}{}}
%\addto\captionsportuguese{\renewcommand{\chaptermark}{}}

%%%% no blank pages between chapters %%%%
\let\cleardoublepage\clearpage


%pacote de modulação
\usepackage{subfiles}

%cabeçalho e rodapé
\usepackage{fancyhdr}
\pagestyle{fancy}
\fancyhead{}
\fancyfoot{}
\fancyhead[L]{Álgebra Linear - Um Livro Colaborativo}
\fancyhead[R]{\thepage}
\fancyfoot[C]{\tiny{Licença CC-BY-SA-3.0. Contato: \url{livroscolaborativos@gmail.com}}}

%%%%%%%%%%%%%%%%%%%%%%%%%%%%%%%%%%%%%%%%%%%%%%%%%%
%   CONVITES À EDIÇÃO
%%%%%%%%%%%%%%%%%%%%%%%%%%%%%%%%%%%%%%%%%%%%%%%%%%

\newcommand{\emconstrucao}{
  \begin{tabular}{|c|}\hline
    Em construção ... Gostaria de participar na escrita deste livro? Veja como em:\\
    \url{https://www.ufrgs.br/reamat/participe.html}\\\hline
  \end{tabular}
}

\newcommand{\construirSec}{
\begin{center}
  Esta seção (ou subseção) está sugerida. Participe da sua escrita. Veja como em:\\
  \url{https://www.ufrgs.br/reamat/participe.html}  
\end{center}
}

\newcommand{\construirExeresol}{
  \begin{center}
    \begin{tabular}{|c|}\hline
      Esta seção carece de exercícios resolvidos. Participe da sua escrita.\\
      Veja como em:\\
      \url{https://www.ufrgs.br/reamat/participe.html}\\\hline
    \end{tabular}
  \end{center}
}

\newcommand{\construirExer}{
  \begin{center}
    \begin{tabular}{|c|}\hline
      Esta seção carece de exercícios. Participe da sua escrita.\\
      Veja como em:\\
      \url{https://www.ufrgs.br/reamat/participe.html}\\\hline
    \end{tabular}
  \end{center}
}

\newcommand{\construirResp}{
\begin{center}
  Este exercício está sem resposta sugerida. Proponha uma resposta. Veja como em:\\
  \url{https://www.ufrgs.br/reamat/participe.html}  
\end{center}
}

%%%%%%%%%%%%%%%%%%%%%%%%%%%%%%%%%%%%%%%%%%%%%%%%%%


\begin{document}

%diretório de origem
\newcommand{\dir}{.} 

\frontmatter

\title{Álgebra Linear\\\small{Um Livro Colaborativo}}
\author{}
\date{\today}

\ifispdf
\addcontentsline{toc}{chapter}{Capa}
\fi

\maketitle

%Este trabalho está licenciado sob a Licença Creative Commons Atribuição-CompartilhaIgual 3.0 Não Adaptada. Para ver uma cópia desta licença, visite https://creativecommons.org/licenses/by-sa/3.0/ ou envie uma carta para Creative Commons, PO Box 1866, Mountain View, CA 94042, USA.

\chapter*{Organizadores}
\addcontentsline{toc}{chapter}{Organizadores}

\begin{itemize}
\item[] Diego Marcon Farias - UFRGS
\item[] Pedro Henrique de Almeida Konzen - UFRGS
\item[] Rafael Rigão Souza - UFRGS
\end{itemize}

%Este trabalho está licenciado sob a Licença Creative Commons Atribuição-CompartilhaIgual 3.0 Não Adaptada. Para ver uma cópia desta licença, visite http://creativecommons.org/licenses/by-sa/3.0/ ou envie uma carta para Creative Commons, PO Box 1866, Mountain View, CA 94042, USA.

%%%%%%%%%%%%%%%%%%%%%%%%%%%%%%%%%%%%%%%
%
% ATENÇÃO
%
%POR SEGURANÇA, NÃO EDITE ESTE ARQUIVO
%
%%%%%%%%%%%%%%%%%%%%%%%%%%%%%%%%%%%%%%%

\chapter*{Colaboradores}
\addcontentsline{toc}{chapter}{Colaboradores}

Este material é fruto da escrita colaborativa. Veja a lista de colaboradores em:
\begin{center}
  \url{https://github.com/reamat/AlgebraLinear/graphs/contributors}
\end{center}

Para saber mais como participar, visite o site oficial do projeto:
\begin{center}
  \url{https://www.ufrgs.br/reamat/AlgebraLinear}
\end{center}
ou comece agora mesmo visitando nosso repositório GitHub:
\begin{center}
  \url{https://github.com/reamat/AlgebraLinear}
\end{center}

%Este trabalho está licenciado sob a Licença Creative Commons Atribuição-CompartilhaIgual 3.0 Não Adaptada. Para ver uma cópia desta licença, visite http://creativecommons.org/licenses/by-sa/3.0/ ou envie uma carta para Creative Commons, PO Box 1866, Mountain View, CA 94042, USA.

\chapter*{Licença}
\addcontentsline{toc}{chapter}{Licença}

Este trabalho está licenciado sob a Licença Creative Commons Atribuição-CompartilhaIgual 3.0 Não Adaptada. Para ver uma cópia desta licença, visite http://creativecommons.org/licenses/by-sa/3.0/ ou envie uma carta para Creative Commons, PO Box 1866, Mountain View, CA 94042, USA.

%Este trabalho está licenciado sob a Licença Creative Commons Atribuição-CompartilhaIgual 3.0 Não Adaptada. Para ver uma cópia desta licença, visite http://creativecommons.org/licenses/by-sa/3.0/ ou envie uma carta para Creative Commons, PO Box 1866, Mountain View, CA 94042, USA.

\chapter*{Nota dos organizadores}
\addcontentsline{toc}{chapter}{Nota dos organizadores}

Nosso objetivo é de fomentar o desenvolvimento de materiais didáticos pela colaboração entre professores e alunos de universidades, institutos de educação e demais interessados no estudo e aplicação da álgebra linear nos mais diversos ramos da ciência e tecnologia.

Para tanto, disponibilizamos em repositório público GitHub (\url{https://github.com/reamat/Calculo}) todo o código-fonte do material em desenvolvimento sob licença Creative Commons Atribuição-CompartilhaIgual 3.0 Não Adaptada (\href{https://creativecommons.org/licenses/by-sa/3.0/}{CC-BY-SA-3.0}). Ou seja, você pode copiar, redistribuir, alterar e construir um novo material para qualquer uso, inclusive comercial. Leia a licença para maiores informações.

O sucesso do projeto depende da colaboração! Participe diretamenta da escrita dos recursos educacionais, dê sugestões ou nos avise de erros e imprecisões. Toda a colaboração é bem vinda. Veja mais sobre o projeto em:
\begin{center}
  \url{https://www.ufrgs.br/reamat/AlgebraLinear}
\end{center}

\vspace{0.5cm}

Desejamos-lhe ótimas colaborações!

%Este trabalho está licenciado sob a Licença Creative Commons Atribuição-CompartilhaIgual 3.0 Não Adaptada. Para ver uma cópia desta licença, visite http://creativecommons.org/licenses/by-sa/3.0/ ou envie uma carta para Creative Commons, PO Box 1866, Mountain View, CA 94042, USA.

\chapter*{Prefácio}
\addcontentsline{toc}{chapter}{Prefácio}

\emconstrucao

\ifispdf
\tableofcontents
\addcontentsline{toc}{chapter}{Sumário}
\fi

\mainmatter

%Inclui os capítulos
\subfile{Semana01/semana01.tex}
\subfile{Semana02/semana02.tex}
\subfile{Semana03/semana03.tex}
\subfile{Semana04/semana04.tex}
\subfile{Semana05/semana05.tex}
\subfile{Semana06-07/semana06-07.tex}

\subfile{Semana09/semana09.tex}
\subfile{Semana10/semana10.tex}
\subfile{Semana11/semana11.tex}
\subfile{Semana12/semana12-fatQR.tex}
\subfile{Semana13/semana13-LLScomQR.tex}
\subfile{Semana14-15/semana14-15.tex}

\backmatter

=======
%Este trabalho está licenciado sob a Licença Creative Commons Atribuição-CompartilhaIgual 3.0 Não Adaptada. Para ver uma cópia desta licença, visite http://creativecommons.org/licenses/by-sa/3.0/ ou envie uma carta para Creative Commons, PO Box 1866, Mountain View, CA 94042, USA.


%%%%%%%%%%%%%%%%%%%%%%%%%%%%%%%%%
%
% ATENÇÃO: NÃO EDITE ESTE ARQUIVO
%
%%%%%%%%%%%%%%%%%%%%%%%%%%%%%%%%%


\documentclass[a4paper]{book}

%%%%%%%%%%%%%%%%%%%%%%%%%%%%%%%%%%%%%%%%%%%%%%
%%%%%%%%Definições para compilação%%%%%%%%%%%%
%%%%%%%%%%%%%%%%%%%%%%%%%%%%%%%%%%%%%%%%%%%%%%
\newif\ifispdf        % O layout será pdf?
\newif\ifishtml       % O layout será html?

\def\tfn{config.knd}   % Arquivo que guarda as definições do tipo de saída
\def \tdata{}          % Definições do tipo de saída: book, slide ou html.

\openin1=\tfn\relax    % Leitura das definições de saída
\read1 to \tdata
\closein1

\tdata                 % Definições de saída

%%%%%%%%%%%%%%%%%%%%%%%%%%%%%%%%%%%%%%%%%%%%%
%%%%%%%%%%%Para escrever português%%%%%%%%%%%
%%%%%%%%%%%%%%%%%%%%%%%%%%%%%%%%%%%%%%%%%%%%%

\usepackage[portuguese]{babel}   %%Portuguese-specific commands
\usepackage[T1]{fontenc}
\usepackage[utf8]{inputenc}



%%%%%%%%%%%%%%%%%%%%%%%%%%%%%%%%%%%%%%%%%%%%%%
%%%%%%%%Pacotes básicos para MathEnvir%%%%%%%%
%%%%%%%%%%%%%%%%%%%%%%%%%%%%%%%%%%%%%%%%%%%%%%

\usepackage{amsmath}%%AMS primary package (includes amstext, amsopn, amsbsy), provides various features for displayed equations and %%other mathematical constructs.

%%%%OPTIONS FOR THE AMSMATH PACKAGE
\usepackage{amscd}     %%Provides a CD environment for simple commutative diagrams (no support for diagonal arrows).
\usepackage{amsxtra}   %%Provides certain odds and ends such as \fracwithdelims and \accentedsymbol, for compatibility with documents %%created using version 1.1.

\usepackage{amsthm}    %%Enhanced version of \newtheorem command for defining theorem-like environments


\usepackage{amssymb}   %%Provides an extended symbol collection (includes amsfonts). For example, \barwedge, \boxdot, \boxminus, %%\boxplus, \boxtimes, \Cap, \Cup (and many more), the arrow \leadsto, and some other symbols such as \Box and \Diamond.

\usepackage{latexsym}  %%makes few additional characters available: \Box \Join \Box \Diamond \leadsto \sqsubset \sqsupset \lhd \unlhd %%\rhd \unrhd

\usepackage[makeroom]{cancel}   % Cancelar termos em equações

\usepackage{enumerate}

%%%%%%%%%%%%%%%%%%%%%%%%%%%%%%%%%%%%%%%%%%%%%%
%%%%%%Color, graphicx, margin (geometry)%%%%%%
%%%%%%%%%%%%%%%%%%%%%%%%%%%%%%%%%%%%%%%%%%%%%%


\ifispdf
\usepackage[dvips]{graphicx}
\fi
\ifishtml
\usepackage[dvips]{graphicx}
\fi

%\usepackage{color}            %%Textcolor, color definitions, etc

%\definecolor{light-blue}{rgb}{0.8,0.85,1}     %%Numbers between 0 and 1
%\definecolor{mygrey}{gray}{0.75}              %%Numbers between 0 and 1

\usepackage{verbatim}         %%Adds text from other files, comment environment

\usepackage{xpatch}           %%Bold theorem titles
\makeatletter
   \xpatchcmd{\@thm}{\fontseries\mddefault\upshape}{}{}{} %same font as thm-header
\makeatother

\usepackage[margin=1in]{geometry}  %%Margins  %%Possible to use \newgeometry to modify small parts mid-document


%font
\ifispdf
\usepackage{tgbonum}
\fi


%%%%%%%%%%%%%%%%%%%%%%%%%%%%%%%%%%%%%%%%%%%%%
%%%%%%%%%%Theorem styles, numbering%%%%%%%%%%
%%%%%%%%%%%%%%%%%%%%%%%%%%%%%%%%%%%%%%%%%%%%%

\theoremstyle{plain}
\newtheorem{theorem}{Teorema}              %%section, chapter, etc
\newtheorem{proposition}[theorem]{Proposição}      %%everything below obeys theorem because: [theorem]
\newtheorem{lemma}[theorem]{Lema}
\newtheorem{corollary}[theorem]{Corolário}
\newtheorem{maintheorem}{Teorema}                   %%number doesn't obey order
\renewcommand{\themaintheorem}{\Alph{maintheorem}}
\newtheorem{maincorollary}{Corolário}
\newtheorem{conjecture}{Conjectura}
\newtheorem*{claim}{Afirmação}

\newtheorem*{teoespectral}{Teorema Espectral}     %%Sem numeração e com o nome desejado

\theoremstyle{definition}
\newtheorem{remark}[theorem]{Observação}
\newtheorem{example}[theorem]{Exemplo}
\newtheorem{definition}[theorem]{Definição}
\newtheorem{exercise}{Exercício}

%%%%%%%%%%%%%%%%%%%%%%%%%%%%%%%%%%%%%%%%%%%%%
%%%%%%%%%%%%%%%%New Commands%%%%%%%%%%%%%%%%%
%%%%%%%%%%%%%%%%%%%%%%%%%%%%%%%%%%%%%%%%%%%%%

%\newcommand{\mb}[1]{\mathbb{#1}}
%\newcommand{\bC}{\mathbb{C}}
%\newcommand{\bE}{\mathbb{E}}
%\newcommand{\bK}{\mathbb{K}}
%\newcommand{\bN}{\mathbb{N}}
%\newcommand{\bP}{\mathbb{P}}
%\newcommand{\bQ}{\mathbb{Q}}
%\newcommand{\bR}{\mathbb{R}}
%\newcommand{\bS}{\mathbb{S}}
%\newcommand{\bT}{\mathbb{T}}
%\newcommand{\bZ}{\mathbb{Z}}

% \newcommand{\mathcal{B}}{\mathcal{B}}
% \newcommand{\cE}{\mathcal{E}}
% \newcommand{\cF}{\mathcal{F}}
% \newcommand{\cH}{\mathcal{H}}
% \newcommand{\cL}{\mathcal{L}}
% \newcommand{\cM}{\mathcal{M}}
% \newcommand{\cO}{\mathcal{O}}
% \newcommand{\cP}{\mathcal{P}}
% \newcommand{\cQ}{\mathcal{Q}}
% \newcommand{\cR}{\mathcal{R}}
% \newcommand{\cS}{\mathcal{S}}

% \newcommand{\al} {\alpha}       \newcommand{\Al}{\Alpha}
% \newcommand{\be} {\beta}        \newcommand{\Be}{\Beta}
% \newcommand{\ga} {\gamma}       \newcommand{\Ga}{\Gamma}
% \newcommand{\de} {\delta}       \newcommand{\De}{\Delta}
% \newcommand{\ep} {\epsilon}
% \newcommand{\eps}{\varepsilon}
% \newcommand{\ze} {\zeta}
% \newcommand{\vte}{\vartheta}
% \newcommand{\iot}{\iota}
% \newcommand{\ka} {\kappa}
% \newcommand{\la} {\lambda}      \newcommand{\La}{\Lambda}
% \newcommand{\vpi}{\varpi}
% %\newcommand{\ro} {\rho}
% \newcommand{\vro}{\varrho}
% \newcommand{\si} {\sigma}       \newcommand{\Si}{\Sigma}
% \newcommand{\vsi}{\varsigma}
% \newcommand{\ups}{\upsilon}     \newcommand{\Up}{\Upsilon}
% \newcommand{\vphi}{\varphi}
% \newcommand{\om} {\omega}       \newcommand{\Om}{\Omega}

% \newcommand{\ang}{\operatorname{angle}}
% \newcommand{\closu}{\operatorname{clos}}
% \newcommand{\Col}{\operatorname{Col}}
% \newcommand{\const}{\operatorname{const}}
% \newcommand{\curl}{\operatorname{curl}}
% \newcommand{\dd}{\, \mathrm{d}}
% \newcommand{\diam}{\operatorname{diam}}
% \newcommand{\Dim}{\operatorname{dim}}
% \newcommand{\Div}{\operatorname{div}}
% \newcommand{\dist}{\operatorname{dist}}
% \newcommand{\grad}{\operatorname{grad}}
% \newcommand{\fr}{\partial}
% \newcommand{\graph}{\operatorname{graph}}
% \newcommand{\id}{\operatorname{Id}}
% \newcommand{\inter}{\operatorname{int}}
% \newcommand{\Leb}{\operatorname{Leb}}
% \newcommand{\length}{\operatorname{length}}
% \newcommand{\Lip}{\operatorname{Lip}}
% \newcommand{\Nul}{\operatorname{Nul}}
% \newcommand{\proj}{\operatorname{proj}}
% \newcommand{\res}{\operatornamewithlimits{Res}}
% \newcommand{\rot}{\operatorname{rot}}
% \newcommand{\var}{\operatornamewithlimits{{var}}}

\newcommand{\proj}{\operatorname{proj}}
\newcommand{\sen}{\operatorname{sen}}
\newcommand{\senh}{\operatorname{senh}}
\newcommand{\Span}{\operatorname{Span}}
\newcommand{\spec}{\operatorname{spec}}
\newcommand{\supp}{\operatorname{supp}}

\everymath{\displaystyle}

%ligações
\usepackage[pdfborder={0 0 0 [0 0]},colorlinks=true,linkcolor=blue,citecolor=blue,filecolor=blue,urlcolor=blue]{hyperref}


%define a marcação dos capítulos
\addto\captionsportuguese{\renewcommand{\chaptername}{}}
%\addto\captionsportuguese{\renewcommand{\chaptermark}{}}

%%%% no blank pages between chapters %%%%
\let\cleardoublepage\clearpage


%pacote de modulação
\usepackage{subfiles}

%cabeçalho e rodapé
\usepackage{fancyhdr}
\pagestyle{fancy}
\fancyhead{}
\fancyfoot{}
\fancyhead[L]{Álgebra Linear - Um Livro Colaborativo}
\fancyhead[R]{\thepage}
\fancyfoot[C]{\tiny{Licença CC-BY-SA-3.0. Contato: \url{livroscolaborativos@gmail.com}}}

%%%%%%%%%%%%%%%%%%%%%%%%%%%%%%%%%%%%%%%%%%%%%%%%%%
%   CONVITES À EDIÇÃO
%%%%%%%%%%%%%%%%%%%%%%%%%%%%%%%%%%%%%%%%%%%%%%%%%%

\newcommand{\emconstrucao}{
  \begin{tabular}{|c|}\hline
    Em construção ... Gostaria de participar na escrita deste livro? Veja como em:\\
    \url{https://www.ufrgs.br/reamat/participe.html}\\\hline
  \end{tabular}
}

\newcommand{\construirSec}{
\begin{center}
  Esta seção (ou subseção) está sugerida. Participe da sua escrita. Veja como em:\\
  \url{https://www.ufrgs.br/reamat/participe.html}  
\end{center}
}

\newcommand{\construirExeresol}{
  \begin{center}
    \begin{tabular}{|c|}\hline
      Esta seção carece de exercícios resolvidos. Participe da sua escrita.\\
      Veja como em:\\
      \url{https://www.ufrgs.br/reamat/participe.html}\\\hline
    \end{tabular}
  \end{center}
}

\newcommand{\construirExer}{
  \begin{center}
    \begin{tabular}{|c|}\hline
      Esta seção carece de exercícios. Participe da sua escrita.\\
      Veja como em:\\
      \url{https://www.ufrgs.br/reamat/participe.html}\\\hline
    \end{tabular}
  \end{center}
}

\newcommand{\construirResp}{
\begin{center}
  Este exercício está sem resposta sugerida. Proponha uma resposta. Veja como em:\\
  \url{https://www.ufrgs.br/reamat/participe.html}  
\end{center}
}

%%%%%%%%%%%%%%%%%%%%%%%%%%%%%%%%%%%%%%%%%%%%%%%%%%

%%%%%%%%%%%%%%%%%%%%%%%%%%%%%%%%%%%%%%%%%%%%%%%%%%
% HTML 
%%%%%%%%%%%%%%%%%%%%%%%%%%%%%%%%%%%%%%%%%%%%%%%%%%
\ifishtml
\renewenvironment{proof}
{{\bfseries\upshape Demonstração.}}
{
  \begin{flushright}
    $\blacksquare$
  \end{flushright}
}
\fi

\usepackage{minitoc}
%%%%%%%%%%%%%%%%%%%%%%%%%%%%%%%%%%%%%%%%%%%%%%%%%%


\begin{document}

%diretório de origem
\newcommand{\dir}{.} 

\frontmatter

\title{Álgebra Linear\\\small{Um Livro Colaborativo}}
\author{}
\date{\today}

\ifispdf
\addcontentsline{toc}{chapter}{Capa}
\fi

\maketitle

%Este trabalho está licenciado sob a Licença Creative Commons Atribuição-CompartilhaIgual 3.0 Não Adaptada. Para ver uma cópia desta licença, visite https://creativecommons.org/licenses/by-sa/3.0/ ou envie uma carta para Creative Commons, PO Box 1866, Mountain View, CA 94042, USA.

\chapter*{Organizadores}
\addcontentsline{toc}{chapter}{Organizadores}

\begin{itemize}
\item[] Diego Marcon Farias - UFRGS
\item[] Pedro Henrique de Almeida Konzen - UFRGS
\item[] Rafael Rigão Souza - UFRGS
\end{itemize}

%Este trabalho está licenciado sob a Licença Creative Commons Atribuição-CompartilhaIgual 3.0 Não Adaptada. Para ver uma cópia desta licença, visite http://creativecommons.org/licenses/by-sa/3.0/ ou envie uma carta para Creative Commons, PO Box 1866, Mountain View, CA 94042, USA.

%%%%%%%%%%%%%%%%%%%%%%%%%%%%%%%%%%%%%%%
%
% ATENÇÃO
%
%POR SEGURANÇA, NÃO EDITE ESTE ARQUIVO
%
%%%%%%%%%%%%%%%%%%%%%%%%%%%%%%%%%%%%%%%

\chapter*{Colaboradores}
\addcontentsline{toc}{chapter}{Colaboradores}

Este material é fruto da escrita colaborativa. Veja a lista de colaboradores em:
\begin{center}
  \url{https://github.com/reamat/AlgebraLinear/graphs/contributors}
\end{center}

Para saber mais como participar, visite o site oficial do projeto:
\begin{center}
  \url{https://www.ufrgs.br/reamat/AlgebraLinear}
\end{center}
ou comece agora mesmo visitando nosso repositório GitHub:
\begin{center}
  \url{https://github.com/reamat/AlgebraLinear}
\end{center}

%Este trabalho está licenciado sob a Licença Creative Commons Atribuição-CompartilhaIgual 3.0 Não Adaptada. Para ver uma cópia desta licença, visite http://creativecommons.org/licenses/by-sa/3.0/ ou envie uma carta para Creative Commons, PO Box 1866, Mountain View, CA 94042, USA.

\chapter*{Licença}
\addcontentsline{toc}{chapter}{Licença}

Este trabalho está licenciado sob a Licença Creative Commons Atribuição-CompartilhaIgual 3.0 Não Adaptada. Para ver uma cópia desta licença, visite http://creativecommons.org/licenses/by-sa/3.0/ ou envie uma carta para Creative Commons, PO Box 1866, Mountain View, CA 94042, USA.

%Este trabalho está licenciado sob a Licença Creative Commons Atribuição-CompartilhaIgual 3.0 Não Adaptada. Para ver uma cópia desta licença, visite http://creativecommons.org/licenses/by-sa/3.0/ ou envie uma carta para Creative Commons, PO Box 1866, Mountain View, CA 94042, USA.

\chapter*{Nota dos organizadores}
\addcontentsline{toc}{chapter}{Nota dos organizadores}

Nosso objetivo é de fomentar o desenvolvimento de materiais didáticos pela colaboração entre professores e alunos de universidades, institutos de educação e demais interessados no estudo e aplicação da álgebra linear nos mais diversos ramos da ciência e tecnologia.

Para tanto, disponibilizamos em repositório público GitHub (\url{https://github.com/reamat/Calculo}) todo o código-fonte do material em desenvolvimento sob licença Creative Commons Atribuição-CompartilhaIgual 3.0 Não Adaptada (\href{https://creativecommons.org/licenses/by-sa/3.0/}{CC-BY-SA-3.0}). Ou seja, você pode copiar, redistribuir, alterar e construir um novo material para qualquer uso, inclusive comercial. Leia a licença para maiores informações.

O sucesso do projeto depende da colaboração! Participe diretamenta da escrita dos recursos educacionais, dê sugestões ou nos avise de erros e imprecisões. Toda a colaboração é bem vinda. Veja mais sobre o projeto em:
\begin{center}
  \url{https://www.ufrgs.br/reamat/AlgebraLinear}
\end{center}

\vspace{0.5cm}

Desejamos-lhe ótimas colaborações!

%Este trabalho está licenciado sob a Licença Creative Commons Atribuição-CompartilhaIgual 3.0 Não Adaptada. Para ver uma cópia desta licença, visite http://creativecommons.org/licenses/by-sa/3.0/ ou envie uma carta para Creative Commons, PO Box 1866, Mountain View, CA 94042, USA.

\chapter*{Prefácio}
\addcontentsline{toc}{chapter}{Prefácio}

\emconstrucao

\ifispdf
\tableofcontents
\addcontentsline{toc}{chapter}{Sumário}
\fi

\mainmatter

%Inclui os capítulos
\subfile{Semana01/semana01.tex}
\subfile{Semana02/semana02.tex}
\subfile{Semana03/semana03.tex}
\subfile{Semana04/semana04.tex}
\subfile{Semana05/semana05.tex}
\subfile{Semana06-07/semana06-07.tex}

\subfile{Semana09/semana09.tex}
\subfile{Semana10/semana10.tex}
\subfile{Semana11/semana11.tex}
\subfile{Semana12/semana12-fatQR.tex}
\subfile{Semana13/semana13-LLScomQR.tex}
\subfile{Semana14-15/semana14-15.tex}

\backmatter

>>>>>>> 5edaf56301b741a287f51829dd7dc48a322ecb58
\end{document}